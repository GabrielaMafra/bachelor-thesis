\chapter{O gerador de código}
\label{cap3}

Dada uma especificação na linguagem \TLA, contendo elementos da lógica TLA e da teoria de conjuntos, além de elementos sintáticos próprios, deseja-se obter uma definição equivalente em linguagem de programação. Equivalência para esse propósito é definida pela igualdade do conjunto de comportamentos permitidos. Isto é, todo comportamento especificado deve ser permitido na execução do código, e todo comportamento permitido pela execução do código deve ter sido especificado.

\section{Elixir}

Para esse propósito, a linguagem de programação escolhida para o código traduzido foi Elixir.

Em vista da relação das especificações em TLA+ e sistemas concorrentes, é interessante que a linguagem do código gerado seja capaz de suportar concorrência em um nível alto de abstração. Adicionalmente, devido à natureza matemática dessas especificações, espera-se minimizar a complexidade e a quantidade de mapeamentos ao traduzi-las para uma linguagem funcional. Ambos estes requistos se fazem necessários pela finalidade de proporcionar um código modificável, de forma que o programador seja capaz de entender a correspondência e minimizando a diferença do nível de abstração no qual ele está programando. Uma linguagem de programação que atende esses requisitos é Elixir.
