\message{ !name(capitulo3.tex)}
\message{ !name(capitulo3.tex) !offset(7) }
\begin{enumerate}
  \item A concorrência é facilitada por ter seu código traduzido para \textit{bytecode} da máquina virtual do Erlang (BEAM). Suporte a concorrência é de extrema importância, já que \TLA foi criado para facilitar a especificação de sistemas concorrentes. É necessário que o código gerado seja capaz de refletir o sistema também nesse quesito.

  \item Uma linguagem funcional tende a se aproximar mais de definições matemáticas do que linguagens de outros paradigmas. Uma vez que a estrutura de \TLA foi construída principalmente no âmbito da matemática, a complexidade das traduções tende a ser menor para uma linguagem funcional.

  \item O alto nível de abstração da sintaxe de Elixir, que se inspira em Ruby e sua busca por código facilmente entendível, faz com o programador que trabalhar com o código gerado possa entendê-lo de forma mais simples e rápida do que seria com uma linguagem de baixo nível. Com isso, otimizações podem ser feitas com mais segurança, e a manutenabilidade do código é favorecida.

  \item A transparência de plataforma provida pela máquina virtual BEAM maximiza o número de ambientes aonde o código pode ser executado. Não seria de muito uso gerar um código para um ambiente específico, e uma máquina virtual permite que o código gerado seja \textit{cross plataform}.

  \item O seu código é aberto sobre a licença Apache 2.0, permitindo que o funcionamento de suas estruturas possa ser verificado a qualquer momento. Não seria possível garantir nenhuma correspondência do código gerado com a especificação se não fosse conhecida a execução gerada pelos operadores usados no código.

\end{enumerate}
\message{ !name(capitulo3.tex) !offset(-15) }
