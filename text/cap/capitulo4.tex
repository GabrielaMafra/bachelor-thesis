\chapter{Considerações}

A etapa inicial do desenvolvimento deste trabalho envolveu, na maior parte, revisão do material bibliográfico e didático disponível sobre TLA e \TLAA. O conhecimento obtido a partir desse estudo foi exposto no Capítulo \ref{cap2}. É esperado que novos conceitos sejam encontrados na continuação do trabalho, o que se transformará em atualizações e adições em tal capítulo. Deseja-se também, com a evolução do conhecimento sobre o assunto, tornar a descrição dos conceitos mais didática e organizada.

Além da apresentação dos conceitos necessários para o entendimento da lógica TLA e da linguagem de especificação \TLAA, foram expostos dois exemplos com a intenção de esclarecer o funcionamento das especificações na linguagem. Esses exemplos são resgatados no Capítulo \ref{cap3} na demonstração de uma instância do objetivo deste trabalho, isto é, numa tradução de especificação para código. Essa demonstração assegura a viabilidade do trabalho, apesar de tratar um escopo restrito do problema. Com o desenvolvimento de mais mapeamentos para compor o tradutor, deseja-se apresentar mais exemplos como o incluso na Seção \ref{mapeamentos}.

Para o exemplo de especificação apresentado na Seção \ref{exemplo2}, deseja-se definir uma outra especificação para exemplificar o conceito de implementação, apresentando um protocolo implementado sobre a especificação apresentada - assim como Lamport demonstra implementação em \cite{video-protocol}. Com esse último exemplo devidamente explicado, é considerado que o Capítulo \ref{cap2} é suficiente para um entendimento dos fundamentos e da aplicação de \TLAA.

Foram justificadas, no Capítulo \ref{cap3}, as escolhas das linguagens Elixir, para o código gerado, e Haskell, para a implementação do tradutor, respectivamente na Seção \ref{elixir} e na Seção \ref{traducao}. Esse mesmo capítulo explica como será composto o tradutor, mostrando a possibilidade de descrevê-lo, neste trabalho e na implementação, através de mapeamentos. O processo de busca por mapeamentos se encontra em um estágio inicial, e deseja-se encontrar mapeamentos que permitam traduzir o maior subconjunto possível de especificações para linguagens, sendo o cenário ideal aquele em que são encontrados mapeamentos suficientes para traduzir corretamente qualquer especificação escrita em \TLAA.

A etapa de implementação do tradutor ainda não foi iniciada. Durante a revisão bibliográfica, contudo, foi encontrada a definição completa da gramática de \TLAA, em \cite{specifying-systems}. Esse artefato facilitará a implementação do \textit{parser}, disponibilizando mais tempo do que o previsto para a implementação dos mapeamentos em si.

Com essas considerações, é evidenciado o progresso deste trabalho durante sua primeira fase, assim como sua viabilidade de conclusão no prazo previsto e a relevância de suas contribuições.

\section{Cronograma}
\label{cronograma}

As etapas para este trabalho foram estabelecidas conforme a lista:

\begin{enumerate}
  \item Etapa 1 - Revisão bibliográfica sobre TLA+: Inclui leitura de textos e tutoriais, busca de exemplos e outras fontes de conhecimento sobre a linguagem, como o curso em vídeo ensinado pelo criador da linguagem.
  \item Etapa 2 - Traduções manuais: Se dá por tentativas de codificação de exemplos de modelos obtidos na Etapa 1. Essa codificação será em uma linguagem de programação funcional com concorrência, maximizando a proximidade com o modelo - possivelmente Elixir.
  \item Etapa 3 - Estabelecimento de mapeamentos: Uma observação das traduções manuais ao lado do código, com objetivo de enumerar mapeamentos feitos no processo manual. Dessa etapa, se espera uma lista de possíveis correspondências a serem avaliadas na Etapa 4 e 5.
  \item Etapa 4 - Escolha de mapeamentos: Dentre os possíveis mapeamentos, serão escolhidos aqueles que apresentam maior correspondência entre as definições, conforme estudos sobre os significados dos construtores utilizados nas duas linguagens.
  \item Etapa 5 - Encontrar garantias para os mapeamentos: O processo de conjecturar, evidenciar ou provar a correspondência entre uma especificação formal qualquer e o código gerado a partir dela com os mapeamentos escolhidos na Etapa 4.
  \item Etapa 6 - Implementação do gerador de código: Consiste em implementar um programa capaz de fazer o \textit{parsing} da linguagem TLA+ e escrever um arquivo com o código na linguagem de programação.
  \item Etapa 7 - Análise de melhorias do ambiente: É o estudo da capacidade de testes para o código gerado manterem a correspondência perante modificações no mesmo. Será feito atráves da implementação de testes e exploração de quais mudanças teriam potencial de tornar o código incoerente com a especificação e passariam nos testes.
\end{enumerate}

\begin{table}[h]
  \centering
  \begin{tabular}{|c||c|c|c|c|c|c|c|c|c|c|c|c|}
  \hline
  \multirow{2}{*}{\textbf{\small{Etapas}}} & \multicolumn{12}{|c||}{\textbf{\small{2019}}} \\
  \cline{2-13}
   & \textbf{J} & \textbf{F} & \textbf{M} & \textbf{A} & \textbf{M} & \textbf{J} & \textbf{J} & \textbf{A} & \textbf{S} & \textbf{O} & \textbf{N} & \textbf{D} \\
  \hline \hline
  \textbf{\small{1}} & & & \cellcolor{gray} & \cellcolor{gray} & \cellcolor{gray} & & & & & & & \\ \hline
  \textbf{\small{2}} & & & & \cellcolor{gray} & & & & & & & & \\ \hline
  \textbf{\small{3}} & & & &  & \cellcolor{gray} & & & & & & & \\ \hline
  \textbf{\small{4}} & & & &  & \cellcolor{gray} & \cellcolor{gray} & \cellcolor{gray} & & & & & \\ \hline
  \textbf{\small{5}} & & & & & & \cellcolor{gray} & \cellcolor{gray} & \cellcolor{gray} & \cellcolor{gray} & & & \\ \hline
  \textbf{\small{6}} & & & & & & \cellcolor{gray} & \cellcolor{gray} & \cellcolor{gray} & \cellcolor{gray} & \cellcolor{gray} & & \\ \hline
  \textbf{\small{7}} & & & & & & & & & \cellcolor{gray} & \cellcolor{gray} & \cellcolor{gray} & \\ \hline
  \end{tabular}
  \caption{Cronograma Proposto}
  \label{tab:cronograma}
\end{table}

O cronograma proposto encontra-se na Tabela \ref{tab:cronograma}. Na primeira fase deste trabalho, contudo, foram finalizadas as etapas 1 e 2, e a etapa 3 está em andamento, juntamente com a 4 e a 5. Assim, o novo cronograma é proposto na Tabela \ref{tab:novo-cronograma}.

\begin{table}[h]
  \centering
  \begin{tabular}{|c||c|c|c|c|c|c|c|c|c|c|c|c|}
  \hline
  \multirow{2}{*}{\textbf{\small{Etapas}}} & \multicolumn{12}{|c||}{\textbf{\small{2019}}} \\
  \cline{2-13}
   & \textbf{J} & \textbf{F} & \textbf{M} & \textbf{A} & \textbf{M} & \textbf{J} & \textbf{J} & \textbf{A} & \textbf{S} & \textbf{O} & \textbf{N} & \textbf{D} \\
  \hline \hline
  \textbf{\small{1}} & & & \cellcolor{gray} & \cellcolor{gray} & \cellcolor{gray} & & & & & & & \\ \hline
  \textbf{\small{2}} & & & & \cellcolor{gray} & & & & & & & & \\ \hline
  \textbf{\small{3}} & & & & & \cellcolor{gray} & \cellcolor{gray} & \cellcolor{gray} & & & & & \\ \hline
  \textbf{\small{4}} & & & & & \cellcolor{gray} & \cellcolor{gray} & \cellcolor{gray} & \cellcolor{gray} & & & & \\ \hline
  \textbf{\small{5}} & & & & & & \cellcolor{gray} & \cellcolor{gray} & \cellcolor{gray} & \cellcolor{gray} & & & \\ \hline
  \textbf{\small{6}} & & & & & & & \cellcolor{gray} & \cellcolor{gray} & \cellcolor{gray} & \cellcolor{gray} & & \\ \hline
  \textbf{\small{7}} & & & & & & & & & \cellcolor{gray} & \cellcolor{gray} & \cellcolor{gray} & \\ \hline
  \end{tabular}
  \caption{Cronograma Atualizado}
  \label{tab:novo-cronograma}
\end{table}
